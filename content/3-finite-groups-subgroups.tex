\section{Terminology and Notation}

\begin{mydef}[Order of a Group]
  The number of elements of a group (finite or infite) is called its \qt{order}. We will use $|G|$ to denote the order of $G$.
\end{mydef}

\begin{mydef}[Order of an Element]
  The \qt{order} of an element $g$ in a group $G$ is the smallest positive integer $n$ such that
  \begin{equation}
    g^n = e $ (In additive notation: $ng = 0$)$
  \end{equation}

  If no such integer $n$ exists, we say that $g$ has \qt{infinite order}. The order of an element is denoted by
  \begin{equation}
    \left | g \right | :\equiv n.
  \end{equation}

\end{mydef}

\begin{mydef}[Subgroup]
  If a subset $H$ of a group $G$ is itself a group under the operation of $G$, we say that $H$ is a \qt{subgroup} of $G$. We use use the notation
  \begin{equation}
    H \leq G.
  \end{equation}

  If we want to indicate, that $H$ is a subgroup of $G$ but is not equal to $G$ itself, we write
  \begin{equation}
    H < G.
  \end{equation}

  Such a subgroup is called \qt{proper subgroup}. The subgroup $\{e\}$ is called \qt{trivial subgroup} of $G$. A subgroup that is not $\{e\}$ is called a \qt{nontrivial subgroup} of $G$.

\end{mydef}

\section{Subgroup Tests}
\begin{thm}[One-Step Subgroup Test]
  \label{thm: One-Step Subgroup Test}
  Leg $G$ be a group and $H$ a nonempty subset of $G$. If $a,b \in H$ $\implies$ $ab^{-1} \in H$,
  then $ H \leq G.$
\end{thm}
\begin{prf}
  Associativity is given by $G$, since the operation of $H$ is the same as that of $G$.

  Now we have to show that the identity $e$ is in $H$. Since $H$ is nonemtpy, we may pick some $x \in H$. Then, letting $a :\equiv x$ and $b :\equiv x$ in the hypothesis, we have $e=xx^{-1} = ab^{-1} \in H$.

  To verify that the inverse $x^{-1} \in H$ whenever $x\in H$, all we need to do is choose $a :\equiv e$ and $b :\equiv x$. If this implies $ab^{-1}=ex^{-1}=x^{-1} \in H$, then every element has an inverse.

  Finally, the proof will be complete when we show that $H$ is closed; that is, if $x,y \in H$, we must show that $xy \in H$ also. So letting
  $a :\equiv x$ and $b :\equiv y^{-1}$, we have
  $ab^{-1} = x(y^{-1})^{-1} = xy \in H.$
\end{prf}
\begin{example}
  Let $G$ be an Abelian group with identity $e$. Let $H :\equiv \{x \in G \mid x^2 = e \}.$ Note that $e^2 = e,$ so $H$ is nonempty.

  Now we assume $a,b \in H$. This means $a^2=e$ and $b^2=e$.
  Finnally, we must show that $(ab^{-1})^2=e$. Since $G$ is Abelian,
  \begin{equation}
    (ab^{-1})^2 = ab^{-1}ab^{-1} = a^2(b^{-1})^2 =a^2(b^2)^{-1}=ee^{-1}=e \in H.
  \end{equation}

  Therefore, $H \leq G$. We have been using the fact that $bb$ is the inverse of $b^{-1}b^{-1}$. [ $(b^2)^{-1} = (b^{-1})^{2}$ ]
\end{example}

\begin{thm}[Two-Step Subgroup Test]
  \label{thm: Two-Step Subgroup Test}
  Let $G$ be a group and let $H$ be a nonempty subset of $G$. If
  \begin{itemize}
    \item $a,b \in H \implies ab \in H$, which means $H$ is closed under the operation and
    \item $a \in H \phantom{,b}\implies a^{-1} \in H$, which means, $H$ is closed under takinging inverses,
  \end{itemize}
  then $H \leq G$.
\end{thm}
\begin{prf}
  Since $H$ is nonempty, the operation of $H$ is associative, $H$ is closed, and every element of $H$ has an inverse in $H$, all that remains is to show that $e$ is in $H$. To this end, let $a$ belong to $H$. Then $a^{-1}$ and $aa^{-1} = e$ are in $H$.
\end{prf}

\setcounter{example}{5}
\begin{example}
  Let $G$ be an Abelian group and let $H :\equiv \{ x \in G \mid \abs{x} \text{ is finite} \}$. Since $e^{1}=e$, $H \neq \{\}$. Let $a,b \in H$ and $m :\equiv \abs{a}$ and $n:\equiv \abs{b}$. Then, because $G$ is Abelian, we have
  \begin{equation}
    (ab)^{mn}=(a^m)^{n} (b^{n})^{m} = e^{n}e^{m}=e.
  \end{equation}
  Thus, $ab$ has a finite order which is not necessarely $mn$. Therefore, $ab \in H$.

  Moreover, $(a^{-1})^{m}=(a^m)^{-1} = e^{-1}=e$ shows, that $a^{-1}$ has finite order and therefore $a^{-1} \in H$. Thus $H \leq G$. \hfill \qedsymbol
\end{example}

\begin{example}
  Let $G$ be an Abelian group and $H \leq G$ and $K \leq G$. Let
  \begin{equation}
    HK :\equiv \{ hk \mid h\in H, k \in K \}.
  \end{equation}

  Note $e =ee \in HK.$ Now suppose $a,b \in HK$ \st
  \begin{equation}
    a=h_1k_1 $ and $ b=h_2k_2, $ where $h_1,h_2 \in H $ and $ k_1, k_2 \in K.
  \end{equation}

  Observe that because $G$ is Abelian, we have
  \begin{equation}
    ab = h_1k_1h_2k_2 = (h_1h_2)(k_1k_2) \in HK.
  \end{equation}

  Likewise,
  \begin{equation}
    a^{-1} = (h_1k_1)^{-1} = k_1^{-1} h_1^{-1} = h_1^{-1} k_1^{-1} \in HK.
  \end{equation}

  So $HK \leq G$. \hfill $\qedsymbol$
\end{example}

How to prove that a subset of a group is \emph{not} a subgroup?
\begin{itemize}
  \item Show that the identity is not in the set.
  \item Exhibit an element in the set whose inverse is not in the set.
  \item Exhibit two elements in the set whose product is not in the set
\end{itemize}

\section{Examples of Subgroups}

\begin{thm}[Finite Subgroup Test]
  \label{thm: Finite Subgroup Test}
  Suppose $H$ is a nonempty finite subset of a group $G$. If $H$ is closed under the operation of $G$, then $H$ is a subgroup of $G$ ($H \leq G$).
\end{thm}
\begin{prf}
  In view of \ref{thm: Two-Step Subgroup Test} (Two-Step Subgroup Test), all that is left to show is that
  \begin{equation}
    a \in H \implies a^{-1} \in H.
  \end{equation}

  If $a=e$, then $a^{-1} = a$ and we are done.

  If $a \neq e$, consider the sequence $a \in H, a^2 \in H, a^3 \in H, \ldots.$ Since $H$ is finite, not all of these elements are distinct. Say $a^i=a^j$ and $i>j$. Then, $a^{i-j} = e$; and since $a\neq e, i-j >1$. Thus, $aa^{i-j-1} = a^{i-j}=e$ and, therefore, $a^{i-j-1}=a^{-1}$ by \ref{thm: Uniqueness of the Inverses}. But $i-j-1 \geq 1$ implies $a^{i-j-1}=a^{-1}\in H$ and we are done.
\end{prf}

\begin{mydef}[notation: $\langle a \rangle$]
  For any element $a$ from a group, let
  \begin{equation}
    \langle a \rangle : \equiv \{ a^{n} \mid n \in \mathbb{Z} \}.
  \end{equation}
\end{mydef}

\begin{thm}[$\langle a \rangle$ is a Subgroup]
  Let $G$ be a group, and let a be any element of $G$. Then, $\langle a \rangle$ is a subgroup of $G$.
\end{thm}
\begin{prf}
  Let $a \in \langle a \rangle \implies \langle a \rangle \neq \{ \}.$ Let $a^n, a^m \in \langle a \rangle$. Then $a^n(a^m)^{-1} = a^{n-m} \in \langle a \rangle$; so, by our One-Step Subgroup Test \ref{thm: One-Step Subgroup Test}, we have that $\langle a \rangle \leq G$.
\end{prf}

The subgroup $\langle a \rangle$ is called the \qt{cyclic subgroup of $G$ generated by $a$}. In case that
\begin{equation}
  G= \langle a \rangle,
\end{equation}
we say that $G$ is \qt{cyclic} and $a$ is a \qt{generator of $G$}. (A cyclic group may have many generators.) Notice that although the list
\begin{equation}
  \ldots, a^{-2}, a^{-1}, a^{0}, a^{-1}, a^{2}, \ldots
\end{equation}

has infinitely many entries, the set
\begin{equation}
  \{ a^{n} \mid n \in \mathbb{Z} \}
\end{equation}

mitght have only finitely many elements. Also note that, since $a^ia^j=a^{i+j}=a^{j+i}=a^ja^i$, every cyclic group is Abelian.

\begin{mydef}[Center of a Group]
  The \qt{center}, $Z(G)$, of a group $G$ is the subset of elements in $G$ that commute with every element in $G$. In symbols,
  \begin{equation}
    Z(G) :\equiv \{a \in G \mid ax = xa \quad \forall x \in G \}.
  \end{equation}

\end{mydef}