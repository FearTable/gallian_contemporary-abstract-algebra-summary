\section{Properties of Cyclic Groups}

%TODO: Figure 4.1

Reminder: A cyclic group $G$ looks as follows
\begin{equation}
  G = \{ a^n \mid n \in \integer \} = \langle a \rangle
\end{equation}

\begin{thm}[Criterion for $a^i = a^j$]
  \label{thm: Criterion for a^i = a^j}
  Let $a \in G$.
  \begin{equation}
    \begin{aligned}
       \abs{a} = \infty & \implies (a^i = a^j \iff i = j ) \\
       \abs{a} = n \;\, & \implies \langle a \rangle = \{e,a,a^2,\ldots, a^{n-1} \} \myand \\
        \quad & \qquad a^i = a^j \iff n \mid (i-j)
    \end{aligned}
  \end{equation}
\end{thm}
\begin{prf}
  \emph{Step 1:} If $a$ has infinite order, no $n$ exists such that $a^n = e$. Since $a^i=a^j \iff a^{i-j} = e$, we must have $i-j=0 \iff i-j = 0$, and the first statement of the theorem is proved.

  \emph{Step 2:} Now we assume $\abs{a} = n$. We will prove that
  \begin{equation}
    \langle a \rangle = \{e, a \ldots, a^{n-1}\}.
  \end{equation}
  Certainly, the elements $e, a, \ldots, a^{n-1}$ are in $\langle a \rangle$.

  Now let $a^k \in \la a \ra.$ By the division algorithm \ref{thm: Divison Algorithm} we have $\exists q,r \in \integer:$
  \begin{equation}
    k=qn+r $ with $ 0 \leq r < n.
  \end{equation}
  \begin{equation}
    \implies a^k=a^{qn+r}=a^{qn}a^r = (a^{n})^q a^r = e^q a^r = a^r
  \end{equation}
  \begin{equation}
    \implies a^k=a^r\in \{e,a,\ldots,a^n\}
  \end{equation}

  This proves that $\la a \ra =  \{e,a,\ldots,a^n\}.$

  \emph{Step 3: } Now assume $a^i = a^j$. $\implies a^{i-j} = e$.

  Again, by the division algorithm \ref{thm: Divison Algorithm} $\implies \exists q,r:$
  \begin{equation}
    i-j = qn + r $ with $ 0 \leq r < n.
  \end{equation}

  Then $a^{i-j}=a^{qn+r}$, and therefore
  \begin{equation}
    e=a^{i-j} = a^{qn+r} = (a^n)^q a^r = e^q a^r = a^r.
  \end{equation}

  Since $n$ is the least positive integer such that $a^n = e$, we must have that $r=0$, so
  \begin{equation}
    n$ divides $i-j.
  \end{equation}

  Conversely (\qt{$\Leftarrow$}), if $i-j=nq$, then $a^{i-j} = a^{nq} =e^q =e$, so that $a^i = a^j$.
\end{prf}

\setcounter{corollary}{0}
\begin{corollary}
  \label{thm: the order of an elmenet of a group equals to the size of the group generated by this element}
  For any group element $a$:
  $\abs{a} = \abs{\la a \ra}$
\end{corollary}
\begin{prf}
  If $\abs{a} = n$, then $\la a \ra = \{e,a,a^2, \ldots, a^{n-1} \}.$ So we have $\abs{\la a \ra} = n = \abs{a}$.
\end{prf}

\begin{corollary}
  \label{thm: the power in which an element equals to the identity is a multiple of the elements order}
  Let $\abs{a}= n$. If $a^k = e,$ then $n \mid k$.
\end{corollary}
\begin{prf}
  $a^k=e=a^0 \iff n \mid (k-0).$
\end{prf}

\begin{corollary}[Relationship between $\abs{ab}$ and $\abs{a}\abs{b}$]
  If $a$ and $b$ belong to a finite group, then
\begin{equation}
    ab = ba \implies \abs{ab} \,\bigg|\, \abs{a} \abs{b}
\end{equation}

\end{corollary}
\begin{prf}
  Let $m :\equiv \abs{a}$ and $b:\equiv \abs{n}$. Then
  \begin{equation}
    (ab)^{mn} = ab\cdot ab\cdots ab= aa \cdot bb \cdots aa \cdot bb =(a^m)^n (b^n)^m = e^ne^m = e.
  \end{equation}
  This implies that $\abs{ab} \bigg | mn$ according to corollary \ref{thm: the power in which an element equals to the identity is a multiple of the elements order}.
\end{prf}

\begin{thm}Let $a\in G$, $n:\equiv \abs{a}=\abs{\la a \ra}$, $k\in \nat^+$. Then
  \begin{equation}
    \la a^k \ra = \la a^{\gcd(n,k)} \ra $ and $ \abs{a ^k} = \frac{n}{\gcd(n,k)}.
  \end{equation}
\end{thm}
\begin{prf} \emph{First part:}
  Let $k\in \nat^+$ and $a\in G$ be given. Let
  \begin{equation}
    n :\equiv \abs{a}
  \end{equation}
  \begin{equation}
    d:\equiv \gcd(n,k) = n\cdot s + k \cdot t, $ for some $s,t \in \integer. $ Let $
  \end{equation}
  \begin{equation}
    r :\equiv \sfrac{k}{d}
  \end{equation}

  Since $a^k = (a^d)^r$, we have by closure that  \begin{equation}
    \label{eq: <a^k> subseteq <a^d>}
    \la a^k \ra \subseteq {\la a^d \ra}.
  \end{equation}

  We also have that
  \begin{equation}
    a^d = a^{ns+kt} = a^{ns}a^{kt} = (a^n)^s (a^k)^t = e (a^k)^t = (a^k)^t \in \la a^k \ra. $ Therefore $
  \end{equation}
  \begin{equation}
    \label{eq: <a^d> subseteq <a^k>}
    \la a^d \ra \subseteq \la a^k \ra.
  \end{equation}
  Equation \eqref{eq: <a^k> subseteq <a^d>} and \eqref{eq: <a^d> subseteq <a^k>} imply that
  \begin{equation}
    \la a^k \ra = \la a^d \ra = \la a^{\gcd(n,k)} \ra
  \end{equation}

  \emph{Second part:} Let $d'$ be any divisor of $n$. Clearly,
  \begin{equation}
    (a^{d'})^{\sfrac{n}{d'}} = a^n =e, $ so that $ \abs{a^{d'}} \leq \sfrac{n}{d'}.
  \end{equation}

  On the other hand, if $i$ is a positive integer less than $\sfrac{n}{d'}$, then
  \begin{equation}
    (a^{d'})^i \neq e $ by the definition of $ \abs{a}=n.
  \end{equation}
  So $\abs{a^{d'}} = \sfrac{n}{d'}.$ Or $ \abs{a^{d'}} = \abs{\la a^{d'} \ra} = \sfrac{n}{d'} = \la a^{\sfrac{n}{d'}} \ra = \abs{a^{\sfrac{n}{d'}}}$  by our corollary  \ref{thm: the order of an elmenet of a group equals to the size of the group generated by this element} from theorem \ref{thm: Criterion for a^i = a^j}.

  We now apply this fact with $d=\gcd(n,k)$ to optain
  \begin{equation}
    \abs{a^k} = \abs{\la a^k \ra} = \abs{ \la a^{\gcd (n,k)} \ra } = \abs{a^{\gcd(n,k)}} = \sfrac{n}{gcd(n,k)}
  \end{equation}
\end{prf}

\setcounter{corollary}{0}

\begin{corollary}[Orders of Elements in a Finite Cyclic Groups]
  In a finite cyclic group, the order of an element divides the order of a group.
\end{corollary}

\begin{corollary}[Criterion for $\la a^i \ra = \la a^j \ra$ and $\abs{a^i} = \abs{a^j}$]
  Let $n:\equiv \abs{a}=\abs{\la a \ra}$. Then
  \begin{equation}
    \la a^i \ra = \la a^j \ra \iff \gcd(n,i) = \gcd(n,j) $ and$
  \end{equation}
  \begin{equation}
    \abs{a^i} = \abs{a^j} \iff \gcd(n,i)=\gcd(n,j).
  \end{equation}
\end{corollary}
\begin{prf}
  Our previous theorem shows us that
  \begin{equation}
    \la a^i \ra = \la a^{\gcd(n,i)}\ra $ and $ \la a^j \ra = \la  a^{\gcd(n,j)} \ra.
  \end{equation}
  Certainly, $\gcd(n,i) = \gcd(n,j)$ implies that
  \begin{equation}
    \la a^{\gcd(n,i)} \ra = \la a^{\gcd(n,j)} \ra
  \end{equation}
  So we have proven the \qt{$\Leftarrow$-direction}.

  On the other hand, $\la a^{\gcd(n,i)} \ra = \la a^{\gcd(n,j)} \ra$ implies that
  \begin{equation}
    \frac{n}{\gcd(n,i)}=\abs{a^{\gcd(n,i)} } = \abs{ a^{\gcd(n,j)} }=\frac{n}{\gcd(n,j)}
  \end{equation}

  by the second conclusion of our previous theorem, and therefore
  \begin{equation}
    \gcd(n,i) = \gcd(n,j)
  \end{equation}

  So we have proven the \qt{$\Rightarrow$-direction}.
\end{prf}

\begin{corollary}[Generators of Finite Cyclic Groups]
  Let $n:\equiv \abs{a}=\abs{\la a \ra}$. Then
  \begin{equation}
    \la a \ra = \la a^j \ra \, \iff \gcd (n,j) = 1, $ and $
  \end{equation}
  \begin{equation}
    \abs{a} = \abs{\la a^j \ra} \iff \gcd (n,j) = 1.
  \end{equation}
\end{corollary}

\begin{corollary}[Generators of $\integer_n$]
  An integer $k$ in $\integer_n$ is a generator of $\integer_n$ $\iff \gcd(n,k) = 1$.
\end{corollary}

\section{Classification of Subgroups of Cyclic Groups}

\begin{thm}[Fundamental Theoreom of Cyclic Groups]
  Every subgroup of a cyclic group is cyclic. Moreover, if $\abs{\la a \ra } = n,$ then the order of any subgroup of $\la a \ra$ is a divisior of $n$; and, for each positive divisor $k$ of $n$, the group $\la a \ra$ has exaclty one subgroup of order $k$ -- namely, $\la a^{n/k} \ra$.
\end{thm}
\begin{prf}
  Let $G :\equiv \la a \ra$ and suppose that $H \leq G$. If $H=\{ e\}$, then clearly $H$ is cyclic. So we may assume $H \neq \{e\}.$ By the definition of $G$ and $H$, $H$ contains an elemnt of the form $a^t$, where $t$ is positive. Because if there is an element $a^k$ where $k$ is negative, $a^{-k}$ also belongs to $H$. Thus our claim is verified. Now let $m$ be the least positive integer such that $a^m \in H$. By closure,
  \begin{equation}
    \label{eq: la am ra subseteq H.}
    \la a^m \ra \subseteq H.
  \end{equation}

  We next claim that $H=\la a^m \ra$. Let $b \in H$. Since $b \in G = \la a \ra$, we have $b = a^k$ for some $k$. Now let $q \in \integer$ and $r \in \nat$ \st
  \begin{equation}
    k = mq + r $ where $ 0 \leq r < m $ by the division algorithm \ref{thm: Divison Algorithm}. Then$
  \end{equation}
  \begin{equation}
    a^k = a^{mq+r} = a^{mq}a^r, $ so that $
  \end{equation}
  \begin{equation}
    a^r=a^{-mq}a^k.
  \end{equation}

  Since $a^k = b \in H$ and $a^{-mq}=(a^m)^{-q} \in H$
  $\implies a^{r} \in H$. But, $m$ is the least positive integer such that $a^m \in H$ and $ 0 \leq r < m $ $\implies r=0$. $\implies k=mq$. Therefore,
  \begin{equation}
    b = a^k = a^{mq} = (a^m)^q \in \la a^m \ra. $ This implies $
  \end{equation}
  \begin{equation}
    H \subseteq \la a^m \ra.$ Together with equation \eqref{eq: la am ra subseteq H.} $ \implies \la a^m \ra = H
  \end{equation}

  This proves the assertion of the theorem that every subgroup of a cyclic group is cyclic.

  To prove the next portion of the theorem, let $n :\equiv \abs{\la a \ra}$ and $H \leq \la a \ra$. Using $b :\equiv a^n = e$ as in the preceding paragraph, we have
  \begin{equation}
    n = mq,
  \end{equation}
  where $m$ is the least positive integer such that $a^m \in H$. Therefore, since we know that $H = \la a^m \ra$, the order of any subgroup of $\la a \ra$ is a divisor of $n$.

  Finally, let $k$ be any positive divisor of $n$. We will show that $\la a^{n/k} \ra$ is the one and only subgroup of $\la a \ra$ of order $k$. We know from our previous theorem that
  \begin{equation}
    \abs{\la a^{n/k} \ra} = n/\gcd(n,n/k) = n/(n/k) =k.
  \end{equation}
  Now let $H$ be any subgroup of $\la a \ra$ of order $k$.
  \begin{equation}
    k :\equiv \abs{\la a \ra}
  \end{equation}
  We have already shown above that $H = \la a^m \ra$, where $m \mid n$.
  \begin{equation}
    \implies m=\gcd(n,m) $ and$
  \end{equation}
  \begin{equation}
    k = \abs{a^m} = \abs{a^{\gcd(n,m)}}=n/\gcd(n,m) = n/m.
  \end{equation}
Thus, $m = \sfrac{n}{k} $ and $ H = \la a^{\sfrac{n}{k}} \ra$.
\end{prf}

\begin{corollary} [Subgroups of $\integer_n$]
  For each positive divisor $k$ of $n$, the set $\la n/k \ra$ is the unique subgroup of $\integer_n$ of order $k$; moreover, these are the only subgroups of $\integer_n$.
\end{corollary}