\section{Properties of Integers}

\begin{mydef}[Well Ordering Principle]
  Every nonempty set of positive integers contains a smallest member. This is called the \qt{Well Ordering Principle}.
\end{mydef}

\begin{mydef}[notation: $a \mid b$]
  For $a,b \in \integer$, $a$ \qt{divides} $b$ $:\iff$ $\exists k\in \integer$ \st
  \begin{equation}
    b=ak.
  \end{equation}

  We write $a \mid b$. Otherwise we write $a \nmid b$.
\end{mydef}

\begin{thm}[Divison Algorithm]
  \label{thm: Divison Algorithm}
  $\forall a,b \in \integer: b > 0 \implies \exists ! q,r \in \integer: $
  \begin{equation}
    a = bq + r $ and $ 0 \leq r < b.
  \end{equation}
\end{thm}
\begin{prf}
  \emph{Existence first:} Let $a,b \in \integer, b > 0$ and also let
  \begin{equation}
    S :\equiv \{ a- bk \mid k \in \integer $ and $ a-bk \geq 0 \}.
  \end{equation}

  If $0 \in S \implies b \mid a $ (because $a=bk$) and we have
  \begin{equation}
    a = b \underbrace{\frac{a}{b}}_{= q} + \underbrace{0.\phantom{\frac{a}{b}\!\!\!\!}}_{= r}
    $ Note that $a$ and $q$ could be $0$ as well$.
  \end{equation}

  Now assume $0 \notin S$. For this case, we first claim that $S$ is nonemtpy.
  \begin{equation}
    a>0 \implies a-0b \in S.
  \end{equation}
  \begin{equation}
    a<0 \implies a-b(2a) = \underbrace{\!\!\phantom{(}a\,\cdot}_{\neq 0} \underbrace{(1-2b)}_{\neq 0} \in S $ because $b$ is positive $ (b>0)
  \end{equation}

  The case where $a=0$ does not exist because $b$ is positive and we assumed that $0 \notin S$.

  Since $S$ is nonempty, we may use the Well Ordering Principle to conclude that $S$ has a smallest member, say
  \begin{equation}
    r = a -bq
  \end{equation}

  for which it holds that
  \begin{equation}
    a=bq+r $ and $ r \geq 0.
  \end{equation}

  All thats left to prove is that $r  < b$. We do this by contradiction. If $r \geq b$
  \begin{equation}
    \implies a-b(q+1) = a-bq -b = r-b \geq 0
  \end{equation}
  \begin{equation}
    \implies a-b(q+1) \in S
  \end{equation}

  But $a-b(q+1) < a-bq$, and $a-bq=r$ is the smallest member in $S$. So $r \geq b$ can not hold. So it must be the case that $r < b.$

  \bigbreak

  \emph{Establishment of uniqueness:} Let $q, \widetilde q,r,\widetilde r \in \integer$ \st
  \begin{equation}
    a = bq+r, \quad 0 \leq r < b
  \end{equation}
  \begin{equation}
    a = b \widetilde q + \widetilde r, \quad 0 \leq \widetilde r < \widetilde b \quad $and$ \quad \widetilde r \geq r.
  \end{equation}

  Surely $bq+r=b \widetilde q+ \widetilde r$ and therefore $b(q-\widetilde q)=\widetilde r-r$. So $b$ devides $\widetilde r-r$ [$b \mid (\widetilde r -r)$] .
  But since we had $r \leq \widetilde r$ we have that
  $0 \leq \widetilde r - r \leq \widetilde r < b$. This and the fact that $b \mid (\widetilde r -r)$ implies that $\widetilde r-r = 0$ which is the same as $\widetilde r=r$. This also implies that $\widetilde q=q$. [because $b \widetilde{q} + 0 = bq + 0.$]

  So we have proven that for every integer $a$ and $b$ if $b >0$, there exists unique integers $q$ and $r$ \st
  $a=bq+r $ and $ 0 \leq r < b.$
\end{prf}

The integer $q$ in the division alogirthm is called the \qt{quotient} upon dividing $a$ by $b$; the integer $r$ is called the \qt{remainder} upon
dividing $a$ by $b$.

\begin{mydef}[Greatest Common Divisor, Relatively Prime Integers]
  The \qt{greatest common divisor} of two nonzero integers $a$ and $b$ is the largest of all common divisors of $a$ and $b$. We denote this integer by $\gcd (a,b)$.

  When $gcd(a,b)=1$, we say $a$, $b$ are \qt{relatively prime}.
\end{mydef}

\begin{thm}[GCD Is a Linear Combination]
  $a,b \in \nat^+ \implies \exists s,t \in \integer$ \st
  \begin{equation}
    \gcd(a,b)=a\cdot s+b \cdot t
  \end{equation}

  Moreover, $\gcd(a,b)$ is the smallest positive integer  of the form $a\cdot s+b \cdot t \in \nat$.
\end{thm}
\begin{prf}
  We first prove the existence of such a linear combination.

  Let $S :\equiv \{am + bn \mid m,n \in \integer $ and $am+bn >0 \}$. We observe that $S$ is nonempty, because $m$ and $n$ can be choosen to make $am + bn$ positive. The Well Ordering Principle asserts that $S$ has a smallest member, say,
  \begin{equation}
    d=as+bt.
  \end{equation}

  We claim that this is our gcd. To verify this claim, we use our Division Algorithm \ref{thm: Divison Algorithm}. So we have
  \begin{equation}
    a = dq + r, $ where $ 0 \leq r < d.
  \end{equation}

  If $r > 0$, then
  \begin{equation}
    r = a-dq = a-(as+bt)q=a-asq-btq=a(1-sq)-b(-tq)\in S.
  \end{equation}

  So $r\in S$, but $r<d$ is contradicting our assumtion that $d$ is the smallest member of $S$. Therefore, $r=0$ and $a=dq$ so $d$ devides $a$. By the same argument, $d$ devides $b$ as well. So $d$ is a common divisor of both $a$ and $b$.

  [we had $0 \leq r < d$ so if $r>0$ does not hold, $r=0$ it the only option]

  \bigbreak

  \emph{Establishment of uniqueness:} Now suppose $d'$ is another common divisor of $a$ and $b$ and write
  \begin{equation}
    a = d'h $ and $ b=d'k $ for some $ h,k \in \nat
  \end{equation}

  Then $d=as + bt = (d'h)s + (d'k)t = (d'h)s + (d'k)t = d'(hs + kt)$, so that $d'$ is a divisor of $d$. Thus among all common divisors of $a$ and $b$, $d$ is the greatest.
\end{prf}
\begin{corollary}
  If $a$ and $b$ are relatively prime, then there exist integers $s$ and $t$ \st
  \begin{equation}
    as + bt = 1.
  \end{equation}
\end{corollary}

\begin{thm-non}[Euclid's Lemma]
  If $p$ is a prime that divides $ab$, then $p$ divides $a$ or $p$ divides $b$. (exclusive \qt{or}, xor)
\end{thm-non}
\begin{prf}
  Suppose $p$ is a prime \st $p \mid ab$ but w.l.o.g. $p \nmid a$.
  \begin{equation}
    \implies 1 =as + pt $ for some $ s,t \in \integer, $ by our Corollary.$
  \end{equation}
  \begin{equation}
    \implies b = bas +bpt = abs + ptb.
  \end{equation}

  Since $p$ divides $abs$ and $ptb$, it follows that $p$ also devides $b$. [and does not divide $a$ according to our assumtion.] The roles of $a$ and $b$ are interchanchable.
\end{prf}

\begin{thm}[Fundamental Theorem of Arithmetic]
  Every integer greater than $1$ is a prime or a product of primes. This product is unique, except fror the order in qhich the factors appear. That is, if
  \begin{equation}
    n = p_1 p_2 \cdots p_r $ and $ n= q_1 q_2 \cdots q_s, $ where the $p$'s and $q$'s are primes, $
  \end{equation}

   then $r=s$ and, after renumbering the $q$'s, we have $p_i = q_i \quad \forall i \in \{1, \ldots, s\}$
\end{thm}